\documentclass[10pt]{article}
\usepackage[usenames]{color} %used for font color
\usepackage{amssymb} %maths
\usepackage{amsmath} %maths
\usepackage[utf8]{inputenc} %useful to type directly diacritic characters
\begin{document}

\documentclass[12pt,a4paper,portrait]{article}

\usepackage{amsfonts}
\usepackage{amsmath}
\usepackage{graphicx}
\usepackage{bm}
\usepackage{mathrsfs}
\usepackage{url}
\usepackage{multirow}
\setlength{\topmargin}{0in}
\setlength{\headheight}{0in}
\setlength{\headsep}{0in}
\setlength{\textheight}{9in}
\setlength{\textwidth}{6.5in}
\setlength{\topskip}{0in}
\setlength{\evensidemargin}{0in}
\setlength{\oddsidemargin}{0in}

\begin{document}
\title{Using the Relaxation Method to solve Poisson's Equation}
\author{Nicole Nikas}
\date{October 16, 2015}
\maketitle
\section{Poisson's Equation in Electrostatics}
Poisson’s Equation for electrostatics is derived using Gauss Law. We start with the divergence of the Electric Field. This is equal to the charge density over the permittivity. The Electric Field is the equal to the negative divergence of the electric potential. Therefore we end up with the divergence of the negative gradient of the electric potential is shown bellow using the bellow equation. 
\begin{equation}
\div \cdot (-div{\phi}) = -\roe/\epsilon
\end{equation}
This can be shown using the Laplace operator as
\begin{equation}
\div^2= = -\roe/\epsilon
\end{equation}
Poisson’s equation is an elliptic partial differential equation of the second form. Therefore the electric potential in two dimensions, can be written as:
(\frac{\delta^2}{\deltax^2}+\frac{\delta^2}{\deltay^2})\phi(x)=f(x,y)

\section{The Relaxation Method}
\section{Using the Relaxation Method to Solve Laplaces Equation}
\section{An Example}
\section{source Code}

\end{document}



\end{document}